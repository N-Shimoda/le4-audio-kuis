\documentclass[a4paper,11pt]{jsarticle}

% 数式
\usepackage{amsmath,amsfonts}
\usepackage{bm}
% 画像
\usepackage[dvipdfmx]{graphicx}
% 枠付き文章
\usepackage{ascmac}
% 位置調整
\usepackage{float}
% ソースコードの表示
\usepackage{listings,jvlisting}
\lstset{
  basicstyle={\ttfamily},
  identifierstyle={\small},
  commentstyle={\smallitshape},
  keywordstyle={\small\bfseries},
  ndkeywordstyle={\small},
  stringstyle={\small\ttfamily},
  frame={tb},
  breaklines=true,
  columns=[l]{fullflexible},
  numbers=left,
  xrightmargin=0zw,
  xleftmargin=3zw,
  numberstyle={\scriptsize},
  stepnumber=1,
  numbersep=1zw,
  lineskip=-0.5ex
}
\renewcommand{\lstlistingname}{ソースコード}


\begin{document}

\title{
  計算機科学実験及演習4	\\   % title
  \large{音響信号処理 レポート1}	% subtitle
}
\author{下田直樹}
\date{提出日: \today}
\maketitle

\tableofcontents
\clearpage

\section{演習2}
\subsection*{あ}
\begin{figure}[H]
  \centering
  \includegraphics[scale=0.5]{../ex2/img/plot-spectrum-2000_a.png}
  \caption{"あ"の振幅スペクトル}
  \label{spectrum_a}
\end{figure}

\subsection*{い}
\begin{figure}[H]
  \centering
  \includegraphics[scale=0.5]{../ex2/img/plot-spectrum-2000_i.png}
  \caption{"い"の振幅スペクトル}
  \label{spectrum_i}
\end{figure}

\subsection*{う}
\begin{figure}[H]
  \centering
  \includegraphics[scale=0.5]{../ex2/img/plot-spectrum-2000_u.png}
  \caption{"う"の振幅スペクトル}
  \label{spectrum_u}
\end{figure}

\subsection*{え}
\begin{figure}[H]
  \centering
  \includegraphics[scale=0.5]{../ex2/img/plot-spectrum-2000_e.png}
  \caption{"え"の振幅スペクトル}
  \label{spectrum_e}
\end{figure}

\subsection*{お}
\begin{figure}[H]
  \centering
  \includegraphics[scale=0.5]{../ex2/img/plot-spectrum-2000_o.png}
  \caption{"お"の振幅スペクトル}
  \label{spectrum_o}
\end{figure}

\section{演習3 : numpy.fft.rfftの動作}
◯◯の定理より(?)任意の信号$x(t)$は、振幅・周波数が異なる複数の正弦波を足し合わせることで次のように表現できる。ここで、$f$は正弦波の周波数であり、$X(f)$は周波数$f$の正弦波の振幅を表す。

$$ x(t) = \int_{-\infty}^{\infty} X(f)e^{2\pi ift} df $$

上式は、$x(t)$が実数値をとる場合のため、積分区間が$-\infty$から$\infty$までの実数値となっている。ここで、$x(t)$がデジタル信号の場合、$x_0, x_1, ..., x_{N-1}$という$N$個の離散的な値をとるため、数列$x_t$と表現できる。

$$ x_t = \sum_{f=0}^{N-1} X_f e^{\frac{2\pi}{N}fit} $$

この周波数$f (f=0,1,...,N-1)$の正弦波の振幅(重み)を求める方法が離散フーリエ変換であり、次式で表せる。

$$ X_f = \frac{1}{\sqrt{N}} \sum_{t=0}^{N-1} x_t e^{\frac{2\pi}{N}ift} $$

\end{document}